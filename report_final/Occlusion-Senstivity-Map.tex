\subsection{Occlusion Sensitivity Maps}
To answer the question of whether the model is really identifying the object while classifying it, this \cite{occlusion} paper uses occlusion maps. This serves to be a coarser technique which does not require backpropagation at all. Here, we perform a systematic occlusion of the input image with an occluding object of some fixed size and monitor the output of the classifier. If and when some main features of the object being identified are occluded, the probability of classification of that object drops sharply. On the other hand, if a competing object is occluded partially or completely, probability of classification increases sharply.

More formally, let our input image be $I \in \mathbb{R}^{m \times n \times c}$. Further, let $p \times q$ be the dimensions of the occlusion. We describe occlusion at the $(j,k)^{th}$ position as
\[I[j:j+p-1][k:k+q-1][:] \leftarrow 0\]
Finally, we prepare a map showing the probability of classification as a function of position of the occluder. This process is justified from natural intuition and it shows
that the visualization genuinely corresponds to the image structure that stimulates that feature map.