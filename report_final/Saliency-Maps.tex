\subsection{Saliency Maps}

Let us review the method used by the authors in the original paper \cite{saliency}. Given a specific class, we query a Convolutional net about the spatial support of that particular class in the image. Formally, let $I$ be the image and let $\mathfrak{S}$ be the image space. We have the vectorizing function
\[L: \mathfrak{S} \to \mathbb{R}^{k\times1}\]
Thus $L(I)$ is the vectorized version of the image. With abuse of notation, let us denote this as $I$. Further for a class $c$ we have the score function
\[S_c: \mathbb{R}^{k\times1} \to \mathbb{R}\]
We consider the linear approximation,
\[S_c(I) \approx w_c^TI + b_c \text{ , where, } w_c^T \in \mathbb{R}^{k\times1}, b_c \in \mathbb{R}\]
Elementary calculus tells us that,
\[w_c = \frac{\delta S_c}{\delta I}\]
Now the authors claim that this matrix gives us an idea of which spatial locations the network is looking at while predicting the class to be $c$. This is backed by the intuition that the larger the derivative for a particular pixel, more the output is sensitive to that pixel. \\ 
Computationally, we arrive at the map as follows. Assume the input shape is $m \times n$. \begin{enumerate}
\item Firstly, the derivative vector is found by backpropagation.
\item Then, the vector is arranged in the same shape as the input image. Denote this $w_{i,j,c}$, where $i \in \{1,2 \dots m\}, j \in \{1,2 \dots n\}$ correspond to the x-y directions, and $c$ to the channel.
\item For all $i \in \{1,2 \dots m\}, j \in \{1,2 \dots n\}$, define $$\mathfrak{M}_{i,j} := max_c|w_{i,j,c}|$$
\end{enumerate}
Then, $\mathfrak{M}$ is the desired saliency map. 
Positive gradients corresponds to the location of target object and Negative image gradients corresponds to the other objects of competing classes. 
Since for the final output we use absolute value, saliency maps are not class discriminative.   
